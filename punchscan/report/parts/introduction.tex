\chapter{Introduction}
\label{chapter:introduction}

In the course of history, many different voting systems have been designed and
used. They differ in their trust models, their usability and other aspects. In
the last decades, ongoing digitalization has also influenced election systems
in that electronic systems have started to play a role in the voting process.

One such system is Punchscan, published in the early 2000s. It is a voting
system using paper ballots, yet heavily relies on cryptographic operations to
provide voter privacy, resilience towards malicious parties, and individual
verifiability. This report aims to provide an overview of Punchscan's
workings and shortcomings.

Chapter \ref{ch:ballot_design_and_voting} will start by discussing the ballot
layout as well as the voting process from the point of view of the voter.
Chapter \ref{ch:setup} will describe the work performed by the election
authority and auditors ahead of time. Chapter \ref{ch:tally} will describe the
tally phase including its audits. Finally chapter \ref{ch:security} will
discuss the integrity and privacy goals --- as well as how and if they are
achieved --- of Punchscan. 

% TODO don't like this sentence
The report is heavily based on the introductory paper to Punchscan by Fisher et
al.\autocite{fisherPunchscanIntroductionSystem2006}, with other work being
referenced where appropriate.

\section{Involved parties}

People participating in the Punchscan election system can be grouped into one
of three categories.
\begin{description}
\item[Voters] Voters participate in the election. They are assumed to be able
to authenticate themselves towards the election authority. Malicious voters may
attempt to learn of others' votes, or sell their own vote to third parties.
\item[Election authority] The election authority consists of a single person
--- or a set of persons in a threshold setting --- which handles the setup and
tally phases of the election. A malicious election authority may attempt to
influence the outcome of the election. In a broader sense the election
authority also includes poll workers assisting voters. The election authority
is assumed to be able to generate key material at will, and share it
securely with other parties where required.
\item[Auditors] Auditors are a set of people tasked with ensuring vote
integrity. At least one auditor is assumed to be honest.
\end{description}

\section{Notation and terminology}

\subsection{Permutations}

Permutations will be denoted by the letter $\pi$, with an index describing
their purpose. As an example, $\pi_{top}$ will be used to refer to the
permutation of the ballot's top page. The standard one-line notation will be
used, where $\pi = cba$ refers to a permutation $\pi$ such that $\pi(a) = c,
\pi(b) = b, \pi(c) = a$. For permutations over two elements the notation from
the Punchscan paper is adopted, where $\rightarrow$ is the identity permutation
and $\circlearrowright$ is the permutation flipping the two elements.
Composition of permutations is denoted as $\pi_1 \circ \pi_2$, evaluated
right-to-left.

\subsection{Representation of data}

The Punchscan system generates and manipulates a fair bit of data, which will
be described in terms of relational databases such as `tables', `rows',
`columns', `primary keys', `foreign keys' and so on. This does not imply a
requirement for specific implementations of Punchscan to use a relational
storage system, although no clear benefit to using non-relational storage is
apparent as the data is inherently relational.
