\chapter{Introduction}
\label{chapter:introduction}

This report provides an overview of the Punchscan voting system. It aims to
explain the utilized concepts, point out shortcomings, and highlight some of
the attacks which have been published since.

% TODO elaborate why we start with this, rather than what is done in setup phase.
Chapter \ref{ch:ballot_design_and_voting} will start by introducing the layout
of the ballot as well as the process of voting from the voter's point of view.

% TODO elaborate on structure of report.

\section{Notation}

\subsection{Permutations}

Permutations will be denoted by the letter $\pi$, with an index describing
their purpose. As an example, $\pi_{top}$ will be used to refer to the
permutation of the ballot's top page. We will use the standard one-line
notation where, for a permutation over elements $\{a, b, c\}$ with the canonical
ordering, $\pi = cba$ refers to a permutation $\pi$ such that $\pi(a) = c$,
$\pi(b) = b$, and $\pi(c) = a$. For a permutation over two elements we also use
the notation as in the paper, where $\rightarrow$ is the identity
permutation and $\circlearrowright$ is the permutation flipping the two
elements. Composition of permutations is denoted as $\pi_1 \circ \pi_2$,
evaluated right-to-left.
