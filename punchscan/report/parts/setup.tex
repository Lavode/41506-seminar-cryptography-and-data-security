\chapter{Setup}
\label{ch:setup}

During the setup phase the election authority will initialize the contents of
three tables. This will be followed by an audit to ensure honesty of the
election authority. The three tables which are initialized are referred to as
the \textbf{P}, \textbf{D} and \textbf{R} tables:
\begin{description}
\item[\emph{P}rint table] The print table contains all information which is
	required to print the ballots, along with information for auditing
		purposes.
\item[\emph{D}ecryption table] The decryption table contains all information
	required to decrypt the voter's encrypted vote in the tally phase,
		along with information for auditing purposes.
\item[\emph{R}esults table] The results table contains the outcome of election.
\end{description}

For the following we will assume an election with one question and two answers
$a$ and $b$, voted on by $n$ voters.

\section{Election authority in a threshold setting}

For the purpose of this chapter we assume the election authority to be a single
entity, in full possession of their private keys. In a real-life deployment it
would be prudent to utilize some form of threshold cryptography to spread this
trust across multiple parties.

\section{Initializing the \textbf{P} table}

The election authority first populates $2n$ rows of the \textbf{P} table as shown
in table \ref{tbl:p_table_full}. This table is indexed by a primary key $ID_P$,
corresponding to the ballot ID which will be printed on both pages of the
ballot. It then picks two random permutations \ptop{} and \pbottom{},
corresponding to the permutations of the top and bottom pages respectively.
Permutations will be shown explicitly. In an actual implementation they might
however be chosen as shown in section \ref{sec:generating_permutations}. The
\emph{Choice} column is left empty, and will be used to store the voter's
permuted choice later on.

For each row it then calculates two cryptographic commitments, \ctop{} and
\cbottom{}, to \ptop{} and \pbottom{} respectively. The utilized commitment
scheme is described in section \ref{sec:commitment_scheme}.

\begin{table}
	\centering
	\begin{tabular}{|c|c|c|c|c|c|}
		\hline
		$ID_P$ & \ptop & \pbottom & $Choice$ & \ctop & \cbottom \\
		\hline
		1 & ab & ab & & $C_{1, 1}$ & $C_{1, 2}$ \\
		2 & ab & ba & & $C_{2, 1}$ & $C_{2, 2}$ \\
		3 & ba & ab & & $C_{3, 1}$ & $C_{3, 2}$ \\
		4 & ba & ba & & $C_{4, 1}$ & $C_{4, 2}$ \\
		5 & ab & ba & & $C_{5, 1}$ & $C_{5, 2}$ \\
		6 & ba & ab & & $C_{6, 1}$ & $C_{6, 2}$ \\
		\hline
	\end{tabular}
	\caption{Print table}
	\label{tbl:p_table_full}
\end{table}

\section{Initializing the \textbf{D} table}

The election authority then populates $2n$ rows of the \textbf{D} table as per
table \ref{tbl:d_table_full}. This table contains a reference to the \textbf{P}
table by means of the $ID_P$ column, and to the \textbf{R} table by means of
the $ID_R$ column. Both $ID_P$ and $ID_R$ are random and independent
permutations of the elements $1, 2, \ldots, 2n$. It then chooses \pone{}
randomly, and calculates \ptwo{} such that $\ptwo \circ \pone \circ \pbottom
\circ \ptop = id$ yields the identity permutation.  The $\hat{R}$ column is
left empty during the setup phase and will be used during decryption. 

Finally a  cryptographic commitment $Com_i$ to each row is generated, as well
as two cryptographic commitments $Com_{ID_P, \pone}$ to the content of columns
$ID_P$ and \pone{}, and $Com_{\ptwo, ID_R}$ to the content of columns \ptwo{}
and $ID_R$.

\begin{table}[h]
	\begin{subtable}{.6\linewidth}
		\centering
		\begin{tabular}{|c|c|c|c|c|c|}
			\hline
			$ID_P$ & $\pi_1$ & $\hat{R}$ & $\pi_2$ & $ID_R$ & $Com_{i}$ \\
			\hline
			6 & $\rightarrow$       & & $\circlearrowright$ & 5 & $C_6$ \\
			5 & $\circlearrowright$ & & $\rightarrow$       & 4 & $C_5$ \\
			2 & $\circlearrowright$ & & $\rightarrow$       & 1 & $C_2$ \\
			1 & $\circlearrowright$ & & $\circlearrowright$ & 3 & $C_1$ \\
			4 & $\rightarrow$       & & $\rightarrow$       & 2 & $C_4$ \\
			3 & $\rightarrow$       & & $\circlearrowright$ & 6 & $C_3$ \\
			\hline
			\multicolumn{2}{|c|}{$Com_{ID_P, \pi_1}$} &   & \multicolumn{2}{c|}{$Com_{\pi_2, ID_R}$} & \\
			\hline
		\end{tabular}
		\caption{Decryption table}
		\label{tbl:d_table_full}
	\end{subtable}%
	\begin{subtable}{.4\linewidth}
		\centering
		\begin{tabular}{|c|c|}
			\hline
			$ID_R$ & $R$ \\
			\hline
			1 & \\
			2 & \\
			3 & \\
			4 & \\
			5 & \\
			6 & \\
			\hline
			\multicolumn{2}{l}{} % Ugly hack to vertically align the two tables.
		\end{tabular}
		\caption{Results table}
		\label{tbl:r_table_full}
	\end{subtable}
	\caption{Decryption and results tables}
\end{table}

\section{Initializing the \textbf{R} table}

In a last step the election authority initializes $2n$ empty rows of the
\textbf{R} table as shown in table \ref{tbl:r_table_full}.

\begin{table}
\end{table}

\section{Setup audit}

The final phase of the setup consists of a first audit by a set of trusted
auditors. The election authority reveals the primary ballot ID $ID_P$ and
commitments of the \textbf{P} table, as well as the row and column commitments
of the \textbf{D} table. The revealed data is shown in table
\ref{tbl:setup_audit}. The auditor then gets to pick $n$ rows at random, for
which the election authority will reveal the full contents, including what is
required to open the commitments. An example of a revealed table is shown in
table \ref{tbl:setup_audit_revealed}. The auditor will then verify that all row
commitments are correct, and that $\ptwo \circ \pone \circ \pbottom \circ \ptop
= id$ holds.

All rows which were fully revealed during this audit are considered spoiled,
and will not be used anymore in subsequent parts of the voting scheme.

\begin{table}[h]
	\centering
	\begin{subtable}{.5\linewidth}
		\begin{tabular}{|c|c|c|c|c|c|}
			\hline
			$ID_P$ & $\pi_{t}$ & $\pi_{b}$ & $c$ & $Com_{\pi_{t}}$ & $Com_{\pi_{b}}$ \\
			\hline
			1 & & & & $C_{1, 1}$ & $C_{1, 2}$ \\
			2 & & & & $C_{2, 1}$ & $C_{2, 2}$ \\
			3 & & & & $C_{3, 1}$ & $C_{3, 2}$ \\
			4 & & & & $C_{4, 1}$ & $C_{4, 2}$ \\
			5 & & & & $C_{5, 1}$ & $C_{5, 2}$ \\
			6 & & & & $C_{6, 1}$ & $C_{6, 2}$ \\
			\hline
		\end{tabular}
	\end{subtable}%
	\begin{subtable}{.5\linewidth}
		\begin{tabular}{|c|c|c|c|c|c|}
			\hline
			$ID_P$ & $\pi_1$ & $\hat{R}$ & $\pi_2$ & $ID_R$ & $Com_{i}$ \\
			\hline
			& & & & & $C_6$ \\
			& & & & & $C_5$ \\
			& & & & & $C_2$ \\
			& & & & & $C_1$ \\
			& & & & & $C_4$ \\
			& & & & & $C_3$ \\
			\hline
			\multicolumn{2}{|c|}{$Com_{ID_P, \pi_1}$} &   & \multicolumn{2}{c|}{$Com_{\pi_2, ID_R}$} & \\
			\hline
		\end{tabular}
	\end{subtable}
	\caption{Subset of tables released for auditing purposes}
	\label{tbl:setup_audit}
\end{table}

\begin{table}
	\centering
	\begin{subtable}{.5\linewidth}
		\begin{tabular}{|c|c|c|c|c|c|}
			\hline
			$ID_P$ & $\pi_{t}$ & $\pi_{b}$ & $c$ & $Com_{\pi_{t}}$ & $Com_{\pi_{b}}$ \\
			\hline
			1 & & & & $C_{1, 1}$ & $C_{1, 2}$ \\
			2 & ab & ba & & $C_{2, 1}$ & $C_{2, 2}$ \\
			3 & & & & $C_{3, 1}$ & $C_{3, 2}$ \\
			4 & ba & ba & & $C_{4, 1}$ & $C_{4, 2}$ \\
			5 & ab & ba & & $C_{5, 1}$ & $C_{5, 2}$ \\
			6 & & & & $C_{6, 1}$ & $C_{6, 2}$ \\
			\hline
		\end{tabular}
	\end{subtable}%
	\begin{subtable}{.5\linewidth}
		\begin{tabular}{|c|c|c|c|c|c|}
			\hline
			$ID_P$ & $\pi_1$ & $\hat{R}$ & $\pi_2$ & $ID_R$ & $Com_{i}$ \\
			\hline
			&                     & &                     &   & $C_6$ \\
			5 & $\circlearrowright$ & & $\rightarrow$       & 4 & $C_5$ \\
			2 & $\circlearrowright$ & & $\rightarrow$       & 1 & $C_2$ \\
			&                     & &                     &   & $C_1$ \\
			4 & $\rightarrow$       & & $\rightarrow$       & 2 & $C_4$ \\
			&                     & &                     &   & $C_3$ \\
			\hline
			\multicolumn{2}{|c|}{$Com_{ID_P, \pi_1}$} &   & \multicolumn{2}{c|}{$Com_{\pi_2, ID_R}$} & \\
			\hline
		\end{tabular}
	\end{subtable}
	\caption{Subset of tables released for auditing purposes}
	\label{tbl:setup_audit_revealed}
\end{table}

\section{Generating permutations}
\label{sec:generating_permutations}

Implementations of Punchscan must be able to generate permutations in such a
way that three properties hold:
\begin{itemize}
	\item Observing parts of a permutation must give an attacker no useful information about the rest of the permutation.
	\item A compact representation must exist, such that it can be stored in a database.
	\item Permutations must be generated computationally, in a way that all
		members of the election authority trust the process.
\end{itemize}

The introductory paper outlines two ways by which such permutations can be
constructed \autocite[section 8]{fisherPunchscanIntroductionSystem2006}. The
first, shown in section \ref{sec:permutations_via_symmetric_cipher} is used to
permute the rows of the $D$ and $R$ tables. The second, shown in section
\ref{sec:permutations_via_shifts} is used to permute the top and bottom pages
of the ballot.

\subsection{Generate permutation over $n$ elements using a symmetric cipher}
\label{sec:permutations_via_symmetric_cipher}

Agree on a symmetric cipher and key $K$. Start with a table with two columns.
Initialize the first column to values $1, 2, \ldots, n$. Fill the second column
with $Enc_K(1), Enc_K(2), \ldots, Enc_K(n)$, where $Enc_K(i)$ is the result of
encrypting $i$ with key $K$ --- using a standard padding scheme if required.
Sort the table by the second column using some canonical ordering. Then, the
order of the numbers in the first column defines a permutation
indistinguishable from a truly random permutation, given standard assumptions
placed on symmetric ciphers.

\subsection{Generate permutation over $n$ elements as combination of two cyclic shifts}
\label{sec:permutations_via_shifts}

For the $\pi_1, \pi_2, \pi_{top}, \pi_{bottom}$ permutations the authors note
that a simpler consruction is sufficient to generate all possible mappings
between answer possibilities and symbols. They propose to generate two random
numbers, and cyclically shift the list of answer possibilities by one of the
random numbers, and the list of answer symbols by the other. This will not
generate all possible permutations as e.g. the relative order of elements is
preserved.

\section{Commitment scheme}

The introductory paper defines their own custom commitment scheme based on the
AES block cipher\autocite[section 9]{fisherPunchscanIntroductionSystem2006}.
Let $K_1, K_2$ be two AES-128 keys. Let $C$ be a public 128-bit constant. Let
$Enc_K(m)$ denote the result of encrypting a 128-bit message $m$ using the key
$K$, $Dec_K(c)$ the result of decrypting a 128-bit ciphertext $c$ using the key
$K$. Let $||$ denote binary concatenation.

\subsection{Key-derivation function}

They first define a custom key-derivation function, shown in algorithm
\ref{alg:kdf}.

\begin{algorithm}
	\begin{algorithmic}
		\State \textbf{Input} $m$
		\State $m_{128} \gets \text{First 128 bits of } m$ \Comment{Pad with zeroes if $m$ shorter than 128 bits}
		\State $K_m \gets Dec_{K_1}(C \oplus Enc_{K_2}(C \oplus Enc_{K_1}(m_{128})))$
		\State \textbf{Return} $K_m$
	\end{algorithmic}
	\caption{$KDF(m)$}
	\label{alg:kdf}
\end{algorithm}

\subsection{Commitment scheme}

They then define a commitment scheme as shown in algorithm \ref{alg:commit}.

\begin{algorithm}
	\begin{algorithmic}
		\State \textbf{Input} $m$
		\State $K_m \gets \Call{KDF}{m}$
		\State $s \gets Enc_{K_m}(C)$
		\State $h_1 \gets \Call{SHA-256}{m || s}$
		\State $h_2 \gets \Call{SHA-256}{m || Enc_{K_m}(h_1)}$ \Comment{AES in ECB mode with PKCS\#5 padding}
		\State \textbf{Return} $(h_1, h_2)$
	\end{algorithmic}
	\caption{$Commit(m)$}
	\label{alg:commit}
\end{algorithm}

\subsection{Opening commitments}

While not described in the paper, one can surmise that opening the commitments
is done by releasing $K_m$. As generation of commitments is deterministic, this
will allow recalculating the commitment and comparing for equality.
