\chapter{Setup}

During the setup phase, the election authority will initialize the contents of
three tables. This will be followed by an audit, to ensure honesty of the
election authority. The three tables which are initialized are referred to as
the \textbf{P}, \textbf{D} and \textbf{R} tables:
\begin{description}
\item[\emph{P}rint table] Contains all information which is required to print
the ballots, along with information for auditing purposes.
\item[\emph{D}ecryption table] Contains all information required to decrypt the
voter's encrypted vote in the tally phase, along with information for auditing
purposes.
\item[\emph{R}esults table] Contains outcome of election.
\end{description}

For the following, we will assume an election with one question, answer
possibilities $a$ and $b$, voted on by $n$ voters.

\section{Election authority in a threshold setting}

For the purpose of this chapter we assume the election authority to be a single entity, in full possession of their private keys. In a real-life deployment it would be prudent to utilize some form of threshold cryptography to spread the trust across multiple parties.

\section{Initializing the \textbf{P} table}

The election authority first populates $2n$ rows of the \emph{P} table as shown
in table . This table is indexed by a primary key $ID_P$,
corresponding to the ballot ID which will be printed on both pages of the
ballot. It then picks two random permutations \ptop{} and \pbottom{}
over $2$ elements, corresponding to the permutations of the top and bottom
pages respectively. Permutations are chosen as described in %TODO
, but will be shown explicitly here.

For each row it then calculates two cryptographic commitments, \ctop{} and
\cbottom{}, to \ptop{} and \pbottom{} respectively.

\begin{table}
\centering
\begin{tabular}{|c|c|c|c|c|c|}
\hline
$ID_P$ & $\pi_{top}$ & $\pi_{bottom}$ & $Choice$ & $Com_{\pi_{top}}$ & $Com_{\pi_{bottom}}$ \\
		\hline
		1 & ab & ab & & $C_{1, 1}$ & $C_{1, 2}$ \\
			2 & ab & ba & & $C_{2, 1}$ & $C_{2, 2}$ \\
			3 & ba & ab & & $C_{3, 1}$ & $C_{3, 2}$ \\
			4 & ba & ba & & $C_{4, 1}$ & $C_{4, 2}$ \\
			5 & ab & ba & & $C_{5, 1}$ & $C_{5, 2}$ \\
			6 & ba & ab & & $C_{6, 1}$ & $C_{6, 2}$ \\
			\hline
			\end{tabular}
			\end{table}
