\chapter{Conclusion}

Punchscan has shown that using cryptographic operations allows one to work with
weaker trust assumptions, such as a potentially malicious election authority.
It has however suffered from a variety of attacks and other weaknesses during
its --- rather short --- life. It thus also serves to highlight the importance
of formalising security proofs and assumptions.

Punchscan did have the advantage that its workings are relatively
straight-forward, and could be explained well to non-technical people. While
more recent electronic voting systems offer more in terms of security against
malicious parties or verifiability, they achieve this by added layers of
complexity which prevent the vast majority of people from getting more than a
cursory understanding of how they work.
